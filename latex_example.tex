\documentclass{book}
\usepackage[utf8]{inputenc}

\usepackage[dvipsnames]{xcolor}
\usepackage{tabularray}

% configuration to make into tbl and longtbl
\SetTblrInner[tblr,longtblr]{
    hlines,
}

\NewTblrEnviron{yukitblr}
% Configs to remove caption and longtables texts
% \DefTblrTemplate{contfoot-text}{yukitblr}{}
% \DefTblrTemplate{conthead-text}{yukitblr}{}
\DefTblrTemplate{caption}{yukitblr}{}
% \DefTblrTemplate{conthead}{yukitblr}{}
% \DefTblrTemplate{capcont}{yukitblr}{}

\SetTblrInner[yukitblr]{
    vlines,
    hlines,
    rowsep=6pt,
    colsep=12pt,
    row{odd} = {LimeGreen!10},
    row{1} = {bg=LimeGreen,fg=white,font=\bfseries},
    cells={c,m},
}
\SetTblrOuter[yukitblr]{
    long,
    label=none,
}

\begin{document}
\chapter{Table example}

Minimum example with tabularray:

\begin{longtblr}[caption={Table caption}]{colspec={X[l]X[c]X[r]}}
    Head 1 & Head 2 & Head 3 \\
    Alpha & Beta & Gamma \\
    Delta & Epsilon & Zeta  \\
    Eta   & Theta   & Iota \\
\end{longtblr}

Another example with custom table definition:


\begin{yukitblr}[caption={Table caption}]{colspec={X[l]X[c]X[r]}}
    Head 1 & Head 2 & Head 3 \\
    Alpha & Beta & Gamma \\
    Delta & Epsilon & Zeta  \\
    Eta   & Theta   & Iota \\
\end{yukitblr}

\end{document}